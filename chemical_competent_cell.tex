%%%%%%%%%%%%%%%%%%%%%%%%%%%%%%%%%%%%%%%%%
% University/School Laboratory Report
% LaTeX Template
% Version 3.1 (25/3/14)
%
% This template has been downloaded from:
% http://www.LaTeXTemplates.com
%
% Original author:
% Linux and Unix Users Group at Virginia Tech Wiki 
% (https://vtluug.org/wiki/Example_LaTeX_chem_lab_report)
%
% License:
% CC BY-NC-SA 3.0 (http://creativecommons.org/licenses/by-nc-sa/3.0/)
%
%%%%%%%%%%%%%%%%%%%%%%%%%%%%%%%%%%%%%%%%%

%----------------------------------------------------------------------------------------
%	PACKAGES AND DOCUMENT CONFIGURATIONS
%----------------------------------------------------------------------------------------

\documentclass[a4paper, twocolumn]{article}

\usepackage[version=3]{mhchem} % Package for chemical equation typesetting
\usepackage{siunitx} % Provides the \SI{}{} and \si{} command for typesetting SI units
\usepackage{graphicx} % Required for the inclusion of images
%\usepackage{natbib} % Required to change bibliography style to APA
\usepackage{amsmath} % Required for some math elements 
\usepackage[a4paper, top=1cm, bottom=2.5cm, left=2.5cm, right=2.5cm, inner=1.5cm, bindingoffset=1cm]{geometry}
\usepackage{booktabs} % for table which have three line type
\setlength\parindent{0pt} % Removes all indentation from paragraphs
\usepackage{float}
\renewcommand{\labelenumi}{\alph{enumi}.} % Make numbering in the enumerate environment by letter rather than number (e.g. section 6)

%\usepackage{times} % Uncomment to use the Times New Roman font
\def\celsius{\ensuremath{^\circ\hspace{-0.09em}\mathrm{C}}}
\newcommand{\tabincell}[2]{\begin{tabular}{@{}#1@{}}#2\end{tabular}} 
\newcommand{\ul}{\mathrm{\mu L}}
%----------------------------------------------------------------------------------------
%	DOCUMENT INFORMATION
%----------------------------------------------------------------------------------------

\title{Recipe: Preparation of  high performance chemical competent Cell} % Title

\author{Pan M. \textsc{chu}} % Author name

\date{\today @ SIAT} % Date for the report

\begin{document}

\maketitle % Insert the title, author and date

\begin{center}
\begin{tabular}{l r}
Date Performed: & August 7, 2019 \\ % Date the experiment was performed
% Partners: & James Smith \\ % Partner names
% & Mary Smith \\
% Instructor: & Professor Fu \\ % Instructor/supervisor
\end{tabular}
\end{center}

% If you wish to include an abstract, uncomment the lines below
% \begin{abstract}
% Abstract text
% \end{abstract}

%----------------------------------------------------------------------------------------
%	SECTION 1
%----------------------------------------------------------------------------------------

% \section{Objective}

% To determine the atomic weight of magnesium via its reaction with oxygen and to study the stoichiometry of the reaction (as defined in \ref{definitions}):

% \begin{center}\ce{2 Mg + O2 -> 2 MgO}\end{center}

% % If you have more than one objective, uncomment the below:
% %\begin{description}
% %\item[First Objective] \hfill \\
% %Objective 1 text
% %\item[Second Objective] \hfill \\
% %Objective 2 text
% %\end{description}

% \subsection{Definitions}
% \label{definitions}
% \begin{description}
% \item[Stoichiometry]
% The relationship between the relative quantities of substances taking part in a reaction or forming a compound, typically a ratio of whole integers.
% \item[Atomic mass]
% The mass of an atom of a chemical element expressed in atomic mass units. It is approximately equivalent to the number of protons and neutrons in the atom (the mass number) or to the average number allowing for the relative abundances of different isotopes. 
% \end{description} 
 
% %----------------------------------------------------------------------------------------
% %	SECTION 2
% %----------------------------------------------------------------------------------------

% \section{Experimental Data}

% \begin{tabular}{ll}
% Mass of empty crucible & \SI{7.28}{\gram}\\
% Mass of crucible and magnesium before heating & \SI{8.59}{\gram}\\
% Mass of crucible and magnesium oxide after heating & \SI{9.46}{\gram}\\
% Balance used & \#4\\
% Magnesium from sample bottle & \#1
% \end{tabular}

% %----------------------------------------------------------------------------------------
% %	SECTION 3
% %----------------------------------------------------------------------------------------

% \section{Sample Calculation}

% \begin{tabular}{ll}
% Mass of magnesium metal & = \SI{8.59}{\gram} - \SI{7.28}{\gram}\\
% & = \SI{1.31}{\gram}\\
% Mass of magnesium oxide & = \SI{9.46}{\gram} - \SI{7.28}{\gram}\\
% & = \SI{2.18}{\gram}\\
% Mass of oxygen & = \SI{2.18}{\gram} - \SI{1.31}{\gram}\\
% & = \SI{0.87}{\gram}
% \end{tabular}

% Because of this reaction, the required ratio is the atomic weight of magnesium: \SI{16.00}{\gram} of oxygen as experimental mass of Mg: experimental mass of oxygen or $\frac{x}{1.31}=\frac{16}{0.87}$ from which, $M_{\ce{Mg}} = 16.00 \times \frac{1.31}{0.87} = 24.1 = \SI{24}{\gram\per\mole}$ (to two significant figures).

% %----------------------------------------------------------------------------------------
% %	SECTION 4
% %----------------------------------------------------------------------------------------

% \section{Results and Conclusions}

% The atomic weight of magnesium is concluded to be \SI{24}{\gram\per\mol}, as determined by the stoichiometry of its chemical combination with oxygen. This result is in agreement with the accepted value.

% \begin{figure}[h]
% \begin{center}
% \includegraphics[width=0.45\textwidth]{placeholder} % Include the image placeholder.png
% \caption{Figure caption.}
% \end{center}
% \end{figure}

% %----------------------------------------------------------------------------------------
% %	SECTION 5
% %----------------------------------------------------------------------------------------

% \section{Discussion of Experimental Uncertainty}

% The accepted value (periodic table) is \SI{24.3}{\gram\per\mole} \cite{Smith:2012qr}. The percentage discrepancy between the accepted value and the result obtained here is 1.3\%. Because only a single measurement was made, it is not possible to calculate an estimated standard deviation.

% The most obvious source of experimental uncertainty is the limited precision of the balance. Other potential sources of experimental uncertainty are: the reaction might not be complete; if not enough time was allowed for total oxidation, less than complete oxidation of the magnesium might have, in part, reacted with nitrogen in the air (incorrect reaction); the magnesium oxide might have absorbed water from the air, and thus weigh ``too much." Because the result obtained is close to the accepted value it is possible that some of these experimental uncertainties have fortuitously cancelled one another.

% %----------------------------------------------------------------------------------------
% %	SECTION 6
% %----------------------------------------------------------------------------------------

% \section{Answers to Definitions}

% \begin{enumerate}
% \begin{item}
% The \emph{atomic weight of an element} is the relative weight of one of its atoms compared to C-12 with a weight of 12.0000000$\ldots$, hydrogen with a weight of 1.008, to oxygen with a weight of 16.00. Atomic weight is also the average weight of all the atoms of that element as they occur in nature.
% \end{item}
% \begin{item}
% The \emph{units of atomic weight} are two-fold, with an identical numerical value. They are g/mole of atoms (or just g/mol) or amu/atom.
% \end{item}
% \begin{item}
% \emph{Percentage discrepancy} between an accepted (literature) value and an experimental value is
% \begin{equation*}
% \frac{\mathrm{experimental\;result} - \mathrm{accepted\;result}}{\mathrm{accepted\;result}}
% \end{equation*}
% \end{item}
% \end{enumerate}

%----------------------------------------------------------------------------------------
%	BIBLIOGRAPHY
%----------------------------------------------------------------------------------------
\section{Introduction}


\section{Protocol}

\begin{itemize}
    \item Streak glycerol stocks in plate with appropriate antibiotics;
    \item Pick one to three colonies into LB medium with appropriate antibiotics;
    \item Overnight culture dilute 100 fold in 50 mL SOB medium with appropriate antibiotics;
    \item Vigorously shaking before $\mathrm{OD_{600}}$ reach to about 0.6 (It is recommended having pre-culture in SOB medium for $\mathrm{OD_{600}}$ reach to 0.3.);
    \item Transform 50 mL medium into 50 mL centrifuge tube and stall it in ice for 10 min;
    \item 4 $\celsius$, 2500 $\mathrm{\times g}$ centrifuge 10 min;
    \item Discard supernatant, re-suspend pellet by 15 mL TB buffer (Table \ref{ta1}.)and stall it in ice for 10 min;
    \item 4 $\celsius$, 2500 $\mathrm{\times g}$ centrifuge 10 min;
    \item Discard supernatant and use 4 mL TB buffer re-suspend pellet, add 300 $\mathrm{\mu L}$ DMSO finally(final concentration: $7\%$ (v/v)).
    \item Add 5 $\ul$  plasmid solution into 100 $\ul$ fresh chemical competent cell.
    \item Stall on ice for 30 min.
    \item 42 $\celsius$ heat-shocked for 45s and chilled on ice for 2 min.
    \item Add 900 $\ul$ SOC medium and shake the culture vigorously.
    \item Coat 100 $\ul$ medium on plate with appropriate antibiotics.
\end{itemize}

\begin{table}[H]
    \centering
    \begin{tabular}{ll}
        \toprule
        Component & Volume \\
        \midrule
        $\mathrm{ddH_2O}$ & 12.5 mL \\
        1 M KCl & 4 mL \\
        0.45M $\mathrm{MnCl_2} $ & 2.4 mL \\
        0.5 M $\mathrm{CaCl_2}$ & 0.6 mL \\
        0.5 M $\textrm{K-MES}^{(1)}$ & 0.5 mL  \\
        \bottomrule
    \end{tabular}
    \caption{20 mL K-MES Buffer Recipe} 
    \label{ta1}
\end{table}
\textbf{Note}: \\
(1) Use KOH adjusting the K-MES solution to pH 6.3, store at $- 20 \celsius$ for long term storage and split into aliquots avoiding repeated freezing and thawing. \\
(2) For a 5 mL system. It can be performed \emph{via} reducing volume of each reagent proportionally. 
\bibliographystyle{ieeetr} 

\bibliography{sample}

%----------------------------------------------------------------------------------------


\end{document}